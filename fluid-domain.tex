\documentclass{standalone}

% put this in your preamble
\usepackage{tikz}
\usetikzlibrary{hobby}

\begin{document}

% put this tikzpicture block in your LaTeX document where you want the figure
\begin{tikzpicture}[use Hobby shortcut,scale=0.5]
  \path
  (0,0) coordinate (z0)
  (10,3) coordinate (z1)
  (4,9) coordinate (z2)
  (1,7) coordinate (z3)
  (2,4) coordinate (z4);
  \draw[closed] (z0) .. (z1) .. (z2) .. (z3) .. (z4);

  % uncomment to show nodes used to construct curve
  %\filldraw (z0) circle (2pt) node[xshift=5mm] {$z_0$};
  %\filldraw (z1) circle (2pt);
  %\filldraw (z2) circle (2pt);
  %\filldraw (z3) circle (2pt);
  %\filldraw (z4) circle (2pt);

  % basic domain labels
  \node at (3,6) {$\Omega$};
  \node[xshift=-1mm,yshift=15mm] at (z1) {$\partial\Omega$};
  \draw[-latex] (z4) node[xshift=-8mm,yshift=2mm] {$\mathbf{n}$} -- (0,3.9);

  % some random arrows
  \draw[-latex] (2,-3) -- +(2,1);
  \draw[-latex] (2.5,3) -- +(1,1);
  \draw[-latex] (4,7) -- +(0.2,1);
  \draw[-latex] (6,-2) -- +(1,1.3);
  \draw[-latex] (6.5,5) -- +(0,1);

  % trajectory ending at x
  \path
  (0.5,0) coordinate (w0)
  (3,1.2) coordinate (w1)
  (4.0,1.2) coordinate (w2)
  (5,2) coordinate (w3);
  \draw[dashed] (w0) .. (w1) .. (w2) .. (w3);
  \node[yshift=-4mm] at (w1) {$X(t)$};
  \filldraw (w3) circle (2.5pt) node[xshift=1.5mm,yshift=-1mm] {$x$};
  \draw[-latex] (w3) node[xshift=10mm,yshift=4mm] {$u(x,t)$} -- +(1,1.5);

\end{tikzpicture}

\end{document}
